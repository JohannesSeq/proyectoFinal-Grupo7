\documentclass[12pt]{article}
\usepackage[utf8]{inputenc}
\usepackage[spanish]{babel}
\usepackage{geometry}
\usepackage{graphicx}
\usepackage{tikz}
\geometry{margin=1in}

\title{Documentación del Proyecto Quinxo}
\author{Grupo 7}
\date{\today}

\begin{document}
\maketitle

\section{Integrantes del Grupo}
\begin{itemize}
    \item Adrián Cordero - Carné: FI21022475 - Git: Adrianca25 - Correo: acordero10422@ufide.ac.cr 
    \item Johannes Sequeira - Carné: FI22023682 - Git: JohannesSeq - Correo: jsequeira40132@ufide.ac.cr 
    \item Kyran Jose Vilchez Barrantes - Carné: FI17008273 - Git: Kvilchezb - Correo: kyran09.97@gmail.com
    \item Jose Daniel Araya Arias - Carné: FI22023911 - Git: pgn3 - Correo: jd2002araya@gmail.com
\end{itemize}

\section{Frameworks y Herramientas}
\begin{itemize}
    \item Frameworks: ASP.NET Core MVC, Entity Framework Core, Bootstrap
    \item Herramientas: SQLite, Visual Studio, GitHub
\end{itemize}

\section{Tipo de Aplicación y Arquitectura}
\begin{itemize}
    \item Tipo de aplicación: MPA (Multi-Page Application)
    \item Arquitectura: MVC (Model-View-Controller)
\end{itemize}

\section{Diagrama de Base de Datos}
\begin{center}
\begin{tikzpicture}[node distance=2cm]
\node (games) [draw, rectangle] {Games | Id:int | Mode:int | CreatedAt:datetime | FinishedAt:datetime | WinnerPlayerId:int | WinnerTeam:string};
\node (players) [draw, rectangle, below of=games] {Players | Id:int | Name:string};
\node (gameplayers) [draw, rectangle, right of=players, xshift=6cm] {GamePlayers | Id:int | GameId:int | PlayerId:int | Order:int | Team:string};
\node (moves) [draw, rectangle, below of=players, yshift=-2cm] {Moves | Id:int | GameId:int | PlayerId:int | FromRow:int | FromCol:int | ToRow:int | ToCol:int | Symbol:string | PointOrientation:string};
\draw[->] (gameplayers.west) -- (players.east);
\draw[->] (gameplayers.north) -- (games.south);
\draw[->] (moves.north) -- (players.south);
\draw[->] (moves.north east) -- (games.south west);
\end{tikzpicture}
\end{center}

\section{Referencias y Prompts IA}
\begin{itemize}
    \item Documentación oficial: ASP.NET Core, Entity Framework, Bootstrap
    \item Prompts IA: Conversaciones con Copilot para revisión de código, generación de README.tex y sugerencias de mejoras
\end{itemize}

\section{Instructivo}
\subsection{Requisitos Previos}
\begin{itemize}
    \item Instalar .NET SDK (versión 6.0 o superior): https://dotnet.microsoft.com/download
    \item Instalar SQLite (incluido en el proyecto, no requiere configuración adicional)
    \item Instalar Entity Framework Core Tools:
    \begin{verbatim}
    dotnet tool install --global dotnet-ef --version 8.0.0
    \end{verbatim}
\end{itemize}

\subsection{Instalación de Dependencias}
Ejecutar en la carpeta del proyecto QuinxoWebApp:
\begin{verbatim}
cd QuinxoWebApp
dotnet restore
\end{verbatim}

\subsection{Compilación}
Compilar el proyecto:
\begin{verbatim}
dotnet build
\end{verbatim}

\subsection{Migraciones y Base de Datos}
Si necesita aplicar migraciones:
\begin{verbatim}
dotnet ef database update
\end{verbatim}

\subsection{Ejecución}
Ejecutar la aplicación:
\begin{verbatim}
dotnet run
\end{verbatim}

Luego abrir en el navegador:
\begin{verbatim}
http://localhost:5212
\end{verbatim}

\subsection{Notas importantes}
\begin{itemize}
    \item Si ejecuta desde la raíz del repositorio, use:
    \begin{verbatim}
    dotnet run --project QuinxoWebApp/QuinxoWebApp.csproj
    \end{verbatim}
    \item Para detener la aplicación, presione Ctrl + C en la consola.
\end{itemize}

\end{document}
