\documentclass[12pt]{article}
\usepackage[utf8]{inputenc}
\usepackage[spanish]{babel}
\usepackage{geometry}
\usepackage{graphicx}
\usepackage{tikz}
\geometry{margin=1in}

\title{Documentación del Proyecto Quinxo}
\author{Grupo 7}
\date{\today}

\begin{document}
\maketitle

\section{Integrantes del Grupo}
\begin{itemize}
    \item Adrián Cordero - Carné: FI21022475 - Git: Adrianca25 - Correo: acordero10422@ufide.ac.cr 
    \item Johannes Sequeira - Carné: FI22023682 - Git: JohannesSeq - Correo: jsequeira40132@ufide.ac.cr 
    \item Kyran Jose Vilchez Barrantes - Carné: FI17008273 - Git: Kvilchezb - Correo: kyran09.97@gmail.com
    \item Jose Daniel Araya Arias - Carné: FI22023911 - Git: pgn3 - Correo: jd2002araya@gmail.com
\end{itemize}

\section{Frameworks y Herramientas}
\begin{itemize}
    \item Frameworks: ASP.NET Core MVC, Entity Framework Core, Bootstrap
    \item Herramientas: SQLite, Visual Studio, GitHub
\end{itemize}

\section{Tipo de Aplicación y Arquitectura}
\begin{itemize}
    \item Tipo de aplicación: MPA (Multi-Page Application)
    \item Arquitectura: MVC (Model-View-Controller)
\end{itemize}

\section{Diagrama de Base de Datos}
\begin{center}
\begin{tikzpicture}[node distance=2cm]
\node (games) [draw, rectangle] {Games | Id:int | Mode:int | CreatedAt:datetime | FinishedAt:datetime | WinnerPlayerId:int | WinnerTeam:string};
\node (players) [draw, rectangle, below of=games] {Players | Id:int | Name:string};
\node (gameplayers) [draw, rectangle, right of=players, xshift=6cm] {GamePlayers | Id:int | GameId:int | PlayerId:int | Order:int | Team:string};
\node (moves) [draw, rectangle, below of=players, yshift=-2cm] {Moves | Id:int | GameId:int | PlayerId:int | FromRow:int | FromCol:int | ToRow:int | ToCol:int | Symbol:string | PointOrientation:string};
\draw[->] (gameplayers.west) -- (players.east);
\draw[->] (gameplayers.north) -- (games.south);
\draw[->] (moves.north) -- (players.south);
\draw[->] (moves.north east) -- (games.south west);
\end{tikzpicture}
\end{center}

\section{Referencias y Prompts IA}

\begin{itemize}

    \item Documentación oficial: ASP.NET Core, Entity Framework y Bootstrap.

    \item Prompts IA: Conversaciones con ChatGPT para revisión de código, generación de \texttt{README.tex} y sugerencias de mejoras.

    \item \textbf{Error 1:} Problema al actualizar el texto de la etiqueta \texttt{turnLabel} en JavaScript.

    \begin{itemize}

        \item Mensaje generado:
\begin{lstlisting}
game.js:297 Uncaught TypeError: Cannot set properties of null (setting 'textContent')
at updateTurnLabel (game.js:297:23)
at HTMLDocument.<anonymous> (game.js:289:5)
\end{lstlisting}

        \item \textbf{Causa:}  
        El JavaScript ejecutaba:
\begin{lstlisting}
document.getElementById("turnLabel").textContent = ...
\end{lstlisting}
        pero el elemento con id \texttt{turnLabel} no existía en la vista, por lo que \texttt{document.getElementById("turnLabel")} devolvía \texttt{null} y acceder a \texttt{null.textContent} provocaba el error.

        \item \textbf{Solución:}  
        Añadir el siguiente elemento en \texttt{Play.cshtml} justo antes del tablero:

\begin{lstlisting}
<h3 id="turnLabel" class="mt-3 mb-3 text-primary" style="font-weight:bold;">
    Turno actual:
</h3>
\end{lstlisting}

        \item Ejemplo completo:
\begin{lstlisting}
<div class="container">
    <h2>Partida #@Model.Id</h2>

    <h3 id="turnLabel" class="mt-3 mb-3 text-primary" style="font-weight:bold;">
        Turno actual:
    </h3>

    <div id="gameTimer" class="timer">00:00:00</div>

    <!-- TABLERO -->
    <div id="boardContainer">
        @* Aquí se renderiza el tablero *@
    </div>

    <button id="resetGame" class="btn btn-danger mt-3">Reiniciar</button>
</div>
\end{lstlisting}

        \item Resultado: El label se actualiza correctamente tras agregar el elemento en la vista.

    \end{itemize}

    \item \textbf{Error 2:} Referencia a variable inexistente \texttt{currentPlayerIndex}.

    \begin{itemize}

        \item Mensaje generado:
\begin{lstlisting}
game.js:294 Uncaught ReferenceError: currentPlayerIndex is not defined
\end{lstlisting}

        \item \textbf{Diagnóstico:}
        \begin{itemize}
            \item El código original dentro de \texttt{updateTurnLabel()} estaba usando \texttt{currentPlayerIndex}.
            \item En el resto del script no existe ninguna variable llamada \texttt{currentPlayerIndex}; en su lugar existe \texttt{currentTurnIndex} (por ejemplo: \texttt{let currentTurnIndex = 0;}).
        \end{itemize}

        \item \textbf{Código con el error (buscar en el archivo):}
\begin{lstlisting}
function updateTurnLabel() {
    const player = PLAYERS[currentPlayerIndex];
    const label = document.getElementById("turnLabel");

    label.textContent = Turno actual: ${player.name} (${player.team});
}
\end{lstlisting}

        \item \textbf{Corrección exacta:} reemplazar \texttt{currentPlayerIndex} por \texttt{currentTurnIndex}:

\begin{lstlisting}
function updateTurnLabel() {
    const player = PLAYERS[currentTurnIndex];
    const label = document.getElementById("turnLabel");

    label.textContent = `Turno actual: ${player.name} (${player.team})`;
}
\end{lstlisting}

        \item \textbf{Nota importante:}
        \begin{itemize}
            \item Verificá que la variable \texttt{currentTurnIndex} esté declarada y actualizada en el scope correcto, por ejemplo:
\begin{lstlisting}
let currentTurnIndex = 0;
let currentPlayer = null;
\end{lstlisting}
            \item Después de la corrección, probá la partida para confirmar que el label muestra el jugador correcto en cada turno.
        \end{itemize}

    \end{itemize}


     \item \textbf{Error 3:}  
    \begin{itemize}
        \item \texttt{Uncaught ReferenceError: PLAYERS is not defined}
        \item \textbf{Solución:}  
        Este error indica que las variables \texttt{PLAYERS}, \texttt{MODE} y \texttt{GAME_ID} no estaban definidas dentro de \texttt{game.js}.  
        Se corrigió asegurando que estas variables fueran enviadas desde la vista mediante un bloque \texttt{<script>} que las expone antes de cargar el archivo JavaScript.
    \end{itemize}

\end{itemize}

\section{Instructivo}
\subsection{Requisitos Previos}
\begin{itemize}
    \item Instalar .NET SDK (versión 6.0 o superior): https://dotnet.microsoft.com/download
    \item Instalar SQLite (incluido en el proyecto, no requiere configuración adicional)
    \item Instalar Entity Framework Core Tools:
    \begin{verbatim}
    dotnet tool install --global dotnet-ef --version 8.0.0
    \end{verbatim}
\end{itemize}

\subsection{Instalación de Dependencias}
Ejecutar en la carpeta del proyecto QuinxoWebApp:
\begin{verbatim}
cd QuinxoWebApp
dotnet restore
\end{verbatim}

\subsection{Compilación}
Compilar el proyecto:
\begin{verbatim}
dotnet build
\end{verbatim}

\subsection{Migraciones y Base de Datos}
Si necesita aplicar migraciones:
\begin{verbatim}
dotnet ef database update
\end{verbatim}

\subsection{Ejecución}
Ejecutar la aplicación:
\begin{verbatim}
dotnet run
\end{verbatim}

Luego abrir en el navegador:
\begin{verbatim}
http://localhost:5212
\end{verbatim}

\subsection{Notas importantes}
\begin{itemize}
    \item Si ejecuta desde la raíz del repositorio, use:
    \begin{verbatim}
    dotnet run --project QuinxoWebApp/QuinxoWebApp.csproj
    \end{verbatim}
    \item Para detener la aplicación, presione Ctrl + C en la consola.
\end{itemize}

\end{document}
